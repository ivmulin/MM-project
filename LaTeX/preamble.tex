% !TEX TS-program = pdflatex
% !TEX encoding = UTF-8 Unicode

\documentclass[11pt]{article}

\usepackage[utf8]{inputenc} % set input encoding (not needed with XeLaTeX)

%%% Examples of Article customizations
% These packages are optional, depending whether you want the features they provide.
% See the LaTeX Companion or other references for full information.

%%% PAGE DIMENSIONS
\usepackage{geometry} % to change the page dimensions
\geometry{a4paper} % or letterpaper (US) or a5paper or....
% \geometry{margin=2in} % for example, change the margins to 2 inches all round
% \geometry{landscape} % set up the page for landscape
%   read geometry.pdf for detailed page layout information

\usepackage{graphicx} % support the \includegraphics command and options

% \usepackage[parfill]{parskip} % Activate to begin paragraphs with an empty line rather than an indent

%%% PACKAGES
\usepackage{booktabs} % for much better looking tables
\usepackage{array} % for better arrays (eg matrices) in maths
\usepackage{paralist} % very flexible & customisable lists (eg. enumerate/itemize, etc.)
\usepackage{verbatim} % adds environment for commenting out blocks of text & for better verbatim
\usepackage{subfig} % make it possible to include more than one captioned figure/table in a single float
\usepackage[hidelinks]{hyperref}
\usepackage{pgfplots}
\usepackage[outline]{contour}
\usepackage{wrapfig}
\usepackage{xcolor}
\usepackage{suffix}
%\usepackage{showframe}

\usepackage[utf8]{inputenc}
\usepackage[T2A]{fontenc}
\usepackage{hyphenat}
\usepackage[russian]{babel}
\usepackage{mathtools}
\usepackage{amsmath}
\usepackage{amsfonts}

% pgfplots settings
\pgfplotsset{width=10cm,compat=1.18}
\usepgfplotslibrary{fillbetween}
\contourlength{0.2em}
% caching pgfplots
%\usepgfplotslibrary{external}
%\tikzexternalize

\hyphenation{
наи-мень-ших стер-жня по-лу-чен-но-го бес-ко-неч-ном теп-ло-про-вод-нос-ти тем-пе-ра-ту-ры дол-жен плот-нос-тью ито-го со-от-но-ше-ние па-ра-мет-ра сле-до-ва-тель-но наи-мень-шую пе-ре-се-че-ния най-ти по-лу-чить най-ден-ной тре-бу-ет-ся гра-ви-та-ци-он-но-го по-ло-жим дос-та-точ-но име-ет ри-сун-ка дос-ти-га-ет-ся зна-чит пря-мая ме-тал-ли-чес-ком за-ви-ся-щая по-строй-те под-чи-ня-ет-ся пре-об-ра-зо-ва-ние
}

%%% HEADERS & FOOTERS
\usepackage{fancyhdr} % This should be set AFTER setting up the page geometry
\pagestyle{fancy} % options: empty , plain , fancy
\renewcommand{\headrulewidth}{0pt} % customise the layout...
\lhead{}\chead{}\rhead{}
\lfoot{}\cfoot{\thepage}\rfoot{}

%%% SECTION TITLE APPEARANCE
\usepackage{sectsty}
\allsectionsfont{\sffamily\mdseries\upshape} % (See the fntguide.pdf for font help)
% (This matches ConTeXt defaults)

%%% ToC (table of contents) APPEARANCE2
\usepackage[nottoc,notlof,notlot]{tocbibind} % Put the bibliography in the ToC
\usepackage[titles,subfigure]{tocloft} % Alter the style of the Table of Contents
\renewcommand{\cftsecfont}{\rmfamily\mdseries\upshape}
\renewcommand{\cftsecpagefont}{\rmfamily\mdseries\upshape} % No bold!

\allowdisplaybreaks % automatically wrap lines in math mode

% colors
\definecolor{pinky}{RGB}{222, 82, 173}
\definecolor{oranguice}{RGB}{227, 140, 86}
\definecolor{bluberry}{RGB}{86, 102, 227}
\definecolor{birchtree}{RGB}{89, 227, 169}

\DeclareMathOperator{\sign}{sgn} % signum - sgn
\newcommand{\Task}[1]{
	\subsection*{\fontfamily{cmss}\selectfont Задача #1.}
	\addcontentsline{toc}{subsection}{Задача #1}
} % task label
\newcommand{\Sln}[1]{
	\subsection*{\fontfamily{cmss}\selectfont Решение.}
	\addcontentsline{toc}{subsection}{Решение задачи #1}
} % solution label
\newcommand{\ScalarProduct}[2]{\langle #1, #2 \rangle} % scalar product
\newcommand{\BigScalarProduct}[2]{\Big \langle #1, #2 \Big \rangle} % scalar product
\newcommand{\Vect}[1]{\mathbf{#1}} % geometric vector
\newcommand{\Fourier}[1]{\mathcal{F} [#1]} % Fourier transform
\newcommand{\bigFourier}[1]{\mathcal{F} \big[ #1 \big]} % Fourier transform for big expressions
\newcommand{\BigFourier}[1]{\mathcal{F} \Big[ #1 \Big]} % Fourier transform for Big expressions
\WithSuffix\newcommand \Fourier*[1]{\mathcal{F}^{-1} [#1]} % inverse Fourier transform
\WithSuffix\newcommand \bigFourier*[1]{\mathcal{F}^{-1} \big[ #1 \big]} % Fourier transform for big expressions
\WithSuffix\newcommand \BigFourier*[1]{\mathcal{F}^{-1} \Big[ #1 \Big]} % Fourier transform for Big expressions
\newcommand{\N}{\mathbb{N}} % Natural numbers - N
\newcommand{\Z}{\mathbb{Z}} % Integer numbers - Z
\newcommand{\Q}{\mathbb{Q}} % Rational numbers - Q
\newcommand{\R}{\mathbb{R}} % Real numbers - R
\newcommand{\Cpx}{\mathbb{C}} % Complex numbers
\newcommand{\e}{\mathrm{e}} % Euler's constant - e
\renewcommand{\d}{\mathrm{d}} % differential - d
\newcommand{\vp}{\operatorname{v. p.}} % principal value - v. p
\newcommand{\erf}{\operatorname{erf}} % error function - erf